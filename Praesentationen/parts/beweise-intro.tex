\begin{frame}{Einführung}
\begin{alertblock}{Was ist ein Beweis?}
\begin{itemize}
        \item lückenlose Folge von logischen Schlüssen,\\welche zur zu beweisenden Behauptung führen
        \item nicht nur einleuchtend, sondern zweifelsfrei korrekt
    \end{itemize}
\end{alertblock}
\end{frame}

\subsubsection{Beweisbeispiel: Transitivität der Teilmenge}
\begin{frame}[fragile]{Beispielbeweis}
\begin{exampleblock}{Zeigen sie, Teilmengen sind transitiv.}
\begin{enumerate}
    \item<1->\alert<1>{
        \only<1>{zu zeigen: }\onslide<2->{z.z. }$A\subseteq B\wedge B\subseteq C \alert<3>{\implies\text{}}A\subseteq C$
        }
    \item<2->\alert<2>{
        \only<2>{Umschreiben:\\}
        $\iff \alert<4,5>{(\alert<9>{(\alert<6>{\forall x}: x \in A \implies x \in B)} \wedge \alert<10>{(\alert<6>{\forall x}: x \in B \implies x \in C)})}$\\
        $\qquad\alert<3>{\implies\text{}}\;(\alert<6>{\forall x}\ : \alert<7>{x \in A} \implies x \in C)$
        }
    \item<3->\alert<3>{
        \only<3>{\emph{Implikation}\\
        linke Seite wahr $\implies$ rechte Seite muss wahr sein.\\
        linke Seite falsch $\implies$ beliebiges kann folgen\\
        $\implies$ uns interessiert also nur der Fall \emph{links ist wahr}}
        \alert<4>{\only<4,5>{Wir machen uns also \emph{\textquotedbl die linke Seite ist wahr\textquotedbl} zur Voraussetzung}\alert<5>{\only<5>{:\\Angenommen, $A \subseteq B \wedge B \subseteq C$ gilt.}}}
        \onslide<6->{Ang., $A \subseteq B \wedge B \subseteq C$.}
        }
    \item<6->\alert<6>{
        \only<6>{Jetzt geht der Beweis richtig los.\\Wähle beliebiges x um Allgemeinheit zu wahren\dots\\}
        \onslide<6->{Sei $x$ beliebig}\alert<7>{\onslide<7->{, mit \alert<9>{$x\in A$.}}}
        }
    \item<8->\alert<8-9>{
        \only<8>{Wir können jetzt unsere Voraussetzungen ausnutzen,\\um $x\in C$ zu folgern.}
        \onslide<9->$\implies x\in B$
        \alert<10>{\onslide<10->$\implies x\in C$}
        \onslide<11>\qed
    }
  \end{enumerate}
\end{exampleblock}
\end{frame}

\begin{frame}[standout]
    Verdauungspause
\end{frame}
