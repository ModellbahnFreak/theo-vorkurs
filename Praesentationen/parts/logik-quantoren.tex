\begin{frame}[fragile]{Quantoren}
    Oft wollen wir Aussagen nicht nur für ein Element, sondern für viele Elemente treffen.
    \metroset{block=fill}
    \begin{exampleblock}{Beispiel}
        $A_1$: Für die Zahl 5 gilt: Sie hat einen Nachfolger\\
        \emph{Allgemeiner:}\\
        $A_2$: Für jede natürliche Zahl n gilt: n hat einen Nachfolger
    \end{exampleblock}
    \begin{exampleblock}{Beispiel}
        $A_3$: Für die Zahl 5 gilt: Sie ist eine Primzahl\\
        \emph{Allgemeiner:}\\
        $A_4$: Es gibt eine natürliche Zahl n, so dass gilt: n ist eine Primzahl
    \end{exampleblock}
\end{frame}

\begin{frame}[fragile]{Quantoren}
    Mithilfe von \textbf{Quantoren} vereinfachen wir uns die Schreibweise dieser Aussagen.\\
    \vspace{0.5cm}
    Quantor \alert{$\forall$}: Die Aussage gilt für alle Elemente.\\
    \metroset{block=fill}
    \begin{exampleblock}{Beispiel}
        $A_1$: $\forall k \in \mathbb{N}:$ 2k ist gerade
    \end{exampleblock}
    Quantor \alert{$\exists$}: Die Aussage gilt für mindestens ein Element.\\
    \metroset{block=fill}
    \begin{exampleblock}{Beispiel}
        $A_2$: $\exists k \in \mathbb{N}:$ k ist Primzahl
    \end{exampleblock}
\end{frame}

\begin{frame}[fragile]{Quantoren}
    In einer Aussage können mehrere Quantoren vorkommen.\\
    Wir lesen dann von links nach rechts.
    \metroset{block=fill}
    \begin{exampleblock}{Beispiel}
        $A_1$: $\forall x,y \in \mathbb{N}: \exists z \in \mathbb{N}: x+y = z$\\
        Bedeutung: Für zwei beliebige Zahlen x und y aus $\mathbb{N}$ gibt es eine weitere natürliche Zahl z, so dass $x+y=z$ gilt.
    \end{exampleblock}
\end{frame}

\begin{frame}[fragile]{Quantoren}
    \alert{Achtung!}\\
    Die Reihenfolge von zwei Quantoren zu vertauschen, kann die Bedeutung einer Aussage deutlich verändern.
    \metroset{block=fill}
    \begin{exampleblock}{Beispiel}
        x,y $\in$ Studenten\\
        \textbf{$A_1$: $\forall x \exists y:$ x schlägt y\\
        $A_2$: $\exists x \forall y:$ x schlägt y\\ }
        Was ist der Unterschied zwischen beiden Aussagen?
    \end{exampleblock}
\end{frame}

\begin{frame}{Quantoren}
    \begin{alertblock}{Aufgabe}
      Wir formulieren folgende Aussage mithilfe von Quantoren und den Symbolen der Aussagenlogik (Junktoren).
    \end{alertblock}
    \metroset{block=fill}
    \begin{block}{Beispiel}
    \begin{itemize}
        \item $A_1$: Eine ganze Zahl ist eine natürliche Zahl, wenn sie positiv oder null ist.
    \end{itemize}
    \end{block}
    \begin{block}{Hinführung}
    \begin{itemize}
        \item $A_1$: Für alle ganzen Zahlen x gilt: Wenn x positiv oder null ist, ist x eine natürliche Zahl.
    \end{itemize}
    \end{block}
    \begin{block}{\alert{Lösung}}
    \begin{itemize}
        \item $A_1$: $\forall x \in \mathbb{Z}: x \geq 0 \implies x \in \mathbb{N}$
    \end{itemize}
    \end{block}
\end{frame}

{\setbeamercolor{palette primary}{bg=ExColor}
\begin{frame}{Denkpause}
    \begin{alertblock}{Aufgaben}
      Formuliere folgende Aussagen mithilfe von Quantoren und den Symbolen der Aussagenlogik (Junktoren). 
    \end{alertblock}
    \metroset{block=fill}
    \begin{block}{Normal}
    \begin{itemize}
        \item $A_1$: Die Differenz zweier ganzer Zahlen ist wieder eine ganze Zahl.
    \end{itemize}
    \end{block}
    \begin{block}{Schwer}
    \begin{itemize}
        \item $A_2$: Jede natürliche Zahl lässt sich als Summe von vier Quadratzahlen darstellen.
    \end{itemize}
    \end{block}
    \begin{block}{Da haben selbst wir keinen Bock drauf}
    \begin{itemize}
        \item $A_3$: Eine natürliche Zahl, die von einer von ihr verschiedenen natürlichen Zahl größer als 1 geteilt wird, ist nicht prim.
    \end{itemize}
    \end{block}
\end{frame}
}

{\setbeamercolor{palette primary}{bg=ExColor}
\begin{frame}{Lösungen}
  \begin{itemize}[<+- | alert@+>]
        \item 
            $A_1$: $\forall x,y \in \mathbb{Z}: x-y \in \mathbb{Z}$
        \item
            $A_2$: $\forall x \in \mathbb{N}: \exists a, b, c, d \in \mathbb{N}: x = a^2 + b^2 + c^2 + d^2$
        \item
            $A_3$: $\forall x \in \mathbb{N}: \left(\exists y \in \mathbb{N}: (y>1) \wedge (y \neq x) \wedge (y \mid x)\right) \implies x\ \text{ist keine Primzahl}$.
    \end{itemize}
\end{frame}
}

\begin{frame}[fragile]{Äquivalente Schreibweisen von Mengenoperationen}
	Oft benötigen wir eine Aussagenlogische Äquivalente Bedingung von Mengenoperationen.
	\metroset{block=fill}
	\begin{block}{Operationen}
		\begin{itemize}
			\item<1-> \textbf{Teilmenge}: A \alert<1>{$\subseteq$} B $\leadsto$ $\forall x \in A: x \in B$\\
			\item<2-> \textbf{Vereinigung}: C = A \alert<2>{$\cup$} B $\leadsto$ $\forall x \in C: x \in A \vee x \in B$\\
			\item<3-> \textbf{Schnitt}: C = A \alert<3>{$\cap$} B $\leadsto$ $\forall x \in C: x \in A \wedge x \in B$\\
			\item<4-> \textbf{Komplement}: \alert<4>{$\overline{A}$} $\leadsto$ $\forall x \in \overline{A}: x \notin A$
		\end{itemize}
	\end{block}
	
\end{frame}
