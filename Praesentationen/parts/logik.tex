\begin{frame}[fragile]{Was ist Aussagenlogik?}
    \begin{alertblock}{Aussagen}
    \begin{itemize}
        \item Paris ist die Hauptstadt von Frankreich
        \item Mäuse jagen Elefanten
        \item $5 \in \mathbb{N}$
        \item 5 = 8
        \item $u \in \{u, v, w\}$
    \end{itemize}
    \end{alertblock}
    Eine Aussage A ist entweder \textbf{wahr} oder \textbf{falsch}.
\end{frame}

\begin{frame}[fragile]{Was ist Aussagenlogik?}
    \begin{alertblock}{Das sind keine Aussagen}
    \begin{itemize}
        \item Macht theoretische Informatik Spaß?
        \item Geh dein Zimmer aufräumen!
        \item Wie viele Tiere wohnen in der Uni?
        \item $(x+y)^2+1$
        \item \{a,b,c\}
        \item ...
    \end{itemize}
    \end{alertblock}
    Diesen Sätzen können wir keinen eindeutigen Wahrheitswert \textbf{wahr} oder \textbf{falsch} zuordnen.
\end{frame}

\begin{frame}[fragile]{Was ist Aussagenlogik?}
    \begin{alertblock}{Wozu brauchen wir das?}
    \begin{itemize}
        \item Wir untersuchen, wie wir Aussagen verknüpfen können.
        \item Damit ziehen wir formale Schlüsse und führen Beweise.
    \end{itemize}
    \end{alertblock}
\end{frame}

\begin{frame}[fragile]{Logische Operationen}
Wir können Aussagen verändern oder durch Operationen zu neuen Aussagen verbinden.
\begin{itemize}
    \item A: Fred möchte Schokolade.
    \item B: Fred möchte Gummibärchen.
\end{itemize}
\metroset{block=fill}
\begin{alertblock}{Grundoperationen}
\begin{itemize}
    \item<1-> \textbf{Und}: A \alert<1>{$\wedge$} B $\leadsto$ Fred möchte Schokolade \alert<1>{und} Gummibärchen.\\
    \only<1>{\emph{Analog}: M: $M_1 \cap M_2$, Jedes Element aus M liegt in $M_1$ \textbf{und} in $M_2$}
    \item<2-> \textbf{Oder}: A \alert<2>{$\vee$} B $\leadsto$ Fred möchte Schokolade \alert<2>{oder} Gummibärchen.\\
    \only<2>{\emph{Anmerkung}: Inklusives \glqq oder\grqq, kein \glqq entweder oder\grqq \\
    Das heißt, es können auch beide Aussagen wahr sein.\\}
	\only<2>{\emph{Analog}: M: $M_1 \cup M_2$, Jedes Element aus M liegt in $M_1$ \textbf{oder} in $M_2$}
    \item<3> \textbf{Nicht}: \alert<3>{$\neg$}A $\leadsto$ Fred möchte \alert<3>{keine} Schokolade.\\
    \only<3>{\emph{Analog}: M: $\overline{M_1}$, Jedes Element aus M liegt \textbf{nicht} in $M_1$}
\end{itemize}
\end{alertblock}
\end{frame}

\begin{frame}{Logische Operationen vs. Mengenoperationen}
	\alert{Mengenoperationen können nicht auf Aussagen angewand werden und umgekehrt!}
	\metroset{block=fill}
	\begin{alertblock}{Grundoperationen}
	\begin{table}[]
		\begin{tabular}{rc|cl}
			\multicolumn{1}{l}{Mengen}             & \multicolumn{1}{l|}{} & \multicolumn{1}{l}{} & \multicolumn{1}{r}{Aussagen}               \\ \hline
			\textbf{Schnitt:}     & A $\cap$ B            & A $\wedge$ B         & \textbf{:Logisches UND}   \\
			\textbf{Vereinigung:} & A $\cup$ B            & A $\vee$ B           & \textbf{:Logisches ODER}  \\
			\textbf{Komplement:}  & $\overline{A}$        & $\neg$ A             & \textbf{:Logisches NICHT}
		\end{tabular}
	\end{table}
	\end{alertblock}
\end{frame}

\begin{frame}{Logische Operationen: Implikation}
\begin{alertblock}{A$\implies$B}
\begin{itemize}
    \item \glqq Wenn A wahr ist, dann muss auch B wahr sein.\grqq
    \item kurz: \glqq\textbf{Wenn} A, \textbf{dann} B.\grqq
    \item Wenn A falsch ist können wir keine Aussage über B treffen.
    \item A$\implies$B ist dieselbe Aussage wie $\neg A \vee B$
\end{itemize}
\end{alertblock}
\end{frame}

\begin{frame}{Logische Operationen: Äquivalenz}
\begin{alertblock}{A$\iff$B}
\begin{itemize}
    \item \glqq A ist wahr, \textbf{genau dann wenn} B wahr ist.\grqq
    \item kurz: \glqq A gdw. B\grqq
    \item A und B müssen den selben Wahrheitswert haben.
    \item A$\iff$B ist dieselbe Aussage wie $(A \implies B) \wedge (B \implies A)$
\end{itemize}
\end{alertblock}
\end{frame}

{\setbeamercolor{palette primary}{bg=ExColor}
\begin{frame}[fragile]{Denkpause}
    \begin{alertblock}{Aufgaben}
      Berechne den Wahrheitswert folgender Aussagen. 
    \end{alertblock}
    \metroset{block=fill}
    \begin{block}{Normal}
    \begin{itemize}
        \item $A_1$: $5 \in \mathbb{N} \wedge a \in \{a, b, c\}$
        \item $A_2$: $0 \in \mathbb{N} \vee a \in \{a, b, c, d\}$
        \item $A_3$: $A_1 \iff A_2$
    \end{itemize}
    \end{block}
    \begin{block}{Etwas Schwerer}
    \begin{itemize}
        \item $A_4$: $(\emptyset=\emptyset^{*}) \implies (a \in \emptyset)$
        \item $A_5$: $\{2\} \cup \{6\} \subseteq \mathbb{N}$
        \item $A_6$: $(a \notin \emptyset) \implies (\emptyset = \emptyset^{*})$
        \item $A_7$: $A_4 \iff A_6$
        \item $A_8$: $(7 \in \{1, 2, 7, 9\}) \cap (2 = 7-5)$
        \item $A_9$: Wenn mein Auto fliegen kann, hast du auch ein fliegendes Auto.
    \end{itemize}
    \end{block}
\end{frame}
}

% {\setbeamercolor{palette primary}{bg=ExColor}
% \begin{frame}[fragile]{Denkpause}
%     \begin{alertblock}{Aufgaben}
%       Löse folgende Zusatzaufgabe. 
%     \end{alertblock}
%     \metroset{block=fill}
%     \begin{block}{Zusatz}
%     \begin{itemize}
%         \item $A_8$: Gegeben zwei Aussagen A, B.\\
%         Formuliere die Aussage $A \iff B$ nur unter Verwendung der Junktoren $\wedge, \vee, \neg$
%     \end{itemize}
%     \end{block}
% \end{frame}
% }
\subsubsection{Aufgaben}
{\setbeamercolor{palette primary}{bg=ExColor}
\begin{frame}{Lösungen}
  \begin{itemize}[<+- | alert@+>]
        \item 
            $A_1$: wahr
        \item
            $A_2$: wahr
        \item
            $A_3$: wahr
        \item
            $A_4$: wahr 
        \item
            $A_5$: wahr
        \item
            $A_6$: falsch
       	\item
       		$A_7$: falsch
        \item
            $A_8$: Das ist keine Aussage, da der Schnitt($\cap$) verwendet wurde um zwei Aussagen miteinander zu verknüpfen
       	\item
       		$A_9$: wahr
    \end{itemize}
\end{frame}
}

\begin{frame}[standout]
    Murmelpause
\end{frame}

% {\setbeamercolor{palette primary}{bg=ExColor}
% \begin{frame}{Lösungen}
%     \metroset{block=fill}
%     \begin{block}{Zusatz}
%     \begin{itemize}
%         \item $A_8$: Gegeben zwei Aussagen A, B.\\
%         Formuliere die Aussage $A \iff B$ nur unter Verwendung der Junktoren $\wedge, \vee, \neg$
%     \end{itemize}
%     \end{block}
%   \begin{alertblock}{Lösung}
%         \begin{itemize}[<+- | alert@+>]
%             \item Äquivalenz bedeutet intuitiv: Beide wahr oder beide falsch
%             \item $A_8$: $(A \wedge B) \vee (\neg A \wedge \neg B)$
%         \end{itemize}
%   \end{alertblock}
% \end{frame}
% }




\begin{frame}{Anwendung der Implikation}
    Wir haben zwei Aussagen A und B. Wir nehmen nun an, A sei wahr. Wenn wir zeigen, dass A$\implies$B wahr ist, wissen wir auch, dass B wahr ist.
\metroset{block=fill}
\begin{exampleblock}{Beispiel}
\begin{enumerate}
    \item<1-> Wir zeigen, es gibt eine ganze Zahl x,\\sodass $3 = x - 2 \implies x = 5$ wahr ist.
    \item<2-> Wir nehmen an, dass die linke Aussage wahr ist\dots
    \item<3-> \dots und zeigen, dass dann die rechte Aussage gilt.
    \item<4-> $(3 = x - 2) \implies (3 + 2 = x) \implies (5 = x) \implies (x = 5)$
    \item<5> Also gilt die rechte Aussage \qed\;
\end{enumerate}
\end{exampleblock}
\end{frame}

% {\setbeamercolor{palette primary}{bg=ExColor}
% \begin{frame}{Denkpause}
%     \begin{alertblock}{Aufgaben}
%       Welche der folgenden Schlüsse sind richtig? 
%     \end{alertblock}
%     \metroset{block=fill}
%     \begin{block}{Normal bis Schwer}
%     \begin{itemize}
%         \item $S_1$: Lukas ist im Vorkurs oder schläft noch. Im Vorkurs ist er nicht. Also schläft er noch.
%         \item $S_2$: Wenn Anne nicht rennt, bekommt sie die Bahn nicht. Sie bekommt die Bahn. Also ist sie gerannt.
%         \item $S_3$: Wenn Tobi nicht lernt, besteht er nicht. Tobi hat gelernt. Also besteht er.
%     \end{itemize}
%     \end{block}
% \end{frame}
% }

% {\setbeamercolor{palette primary}{bg=ExColor}
% \begin{frame}{Lösungen}
%   \begin{itemize}[<+- | alert@+>]
%         \item 
%             $S_1$: Richtig.
%         \item
%             $S_2$: Richtig. Begründung: Die zweite Aussage ist die Kontraposition der ersten Aussage. Wie gezeigt wurde, ist die Kontraposition einer Aussage wahr gdw. die Aussage selbst wahr ist.
%         \item
%             $S_3$: Falsch. Beispiel: Tobi lernt, und besteht trotzdem nicht. \Frowny
%     \end{itemize}
% \end{frame}
% }
