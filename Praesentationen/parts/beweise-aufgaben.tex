%\subsubsection{Aufgaben}
{\setbeamercolor{palette primary}{bg=ExColor}
\begin{frame}[fragile]{Denkpause}
    \begin{alertblock}{Aufgaben}
    Welche Beweistechnik könnte sich für die folgenden Aussagen eignen? Warum?
    \end{alertblock}
    
    \metroset{block=fill}
    \begin{block}{Normal}
        \begin{itemize}
            \item Für jede Primzahl $p$ ist $2^p-1$ eine Primzahl.
        \end{itemize}
    \end{block}
    \metroset{block=fill}
    \begin{block}{Etwas schwerer}
    \begin{itemize}
            \item Für jede ganze Zahl $x$ gilt $x\equiv 1\bmod 4 \implies x\equiv 1\bmod 2$
    \end{itemize}
        
    \end{block}
\end{frame}
}

{\setbeamercolor{palette primary}{bg=ExColor}
\begin{frame}[fragile]{Lösungen}
    \begin{alertblock}{Aufgaben}
    Welche Beweistechnik könnte sich für die folgenden Aussagen eignen? Warum?
    \end{alertblock}
    
    \metroset{block=fill}
    \begin{block}{Normal}
        \begin{itemize}
            \item Für jede Primzahl $p$ ist $2^p-1$ eine Primzahl.\\
            $\rightarrow$ Gegenbeispiel (sei $p\defeq11$)
        \end{itemize}
    \end{block}
    \metroset{block=fill}
    \begin{block}{Etwas schwerer}
        \begin{itemize}
            \item[] $\rightarrow$Direkter Beweis
            \item $x\equiv 1\bmod 4\iff\exists z\in\mathbb{Z} : 4 \cdot z + 1 = x$
            \item $\iff\exists z\in\mathbb{Z}: (2 \cdot 2) \cdot z + 1 = x$
            \item $\implies\exists u,z\in\mathbb{Z}: u = 2z\wedge 2u + 1 = x$
            \item $\implies x\equiv 1\bmod 2$
        \end{itemize}
    \end{block}
\end{frame}
}
