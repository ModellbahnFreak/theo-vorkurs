\begin{frame}[fragile]{Wie sprechen wir das?}
$L_2 = \{a^n \alert{\mid} n \in \mathbb{N}\}$ \\

\emph{Die Sprache $L_2$ enthält alle Wörter $a^n$, \alert{für die gilt}: n stammt aus der Menge der natürlichen Zahlen.}
\vspace{5pt}

\metroset{block=fill}
\begin{alertblock}{Achtung}
    In der theoretischen Informatik enthält $\mathbb{N}$ ($\mathbb{N}$ ist die Menge der natürlichen Zahlen) die Zahl 0.
\end{alertblock}

\end{frame}

\begin{frame}[fragile]{Wie schreiben wir das?}
    \begin{itemize}
        \item
        Viele Zeichen hintereinander (konkateniert) können auch einfacher geschrieben werden.
        \begin{align*}
            a^0 &= \emptyWord\\
            a^1 &= a\\
            a^2 &= a \cdot a = aa\\
            a^3 &= a \cdot a \cdot a = aaa\\
            &\vdots\\
            a^n &= \underbrace{a \cdot a \cdot ... \cdot a}_{\text{n-mal}}
        \end{align*}
        
        \end{itemize}
\end{frame}

\begin{frame}[fragile]{Wie schreiben wir das?}
    \begin{itemize}[<+- | alert@+>]
        \item $L_1 = \{0,2,4,6,8,\dots\} = \{x \mid \text{x ist gerade}\}$\\
        \hspace{4.5mm}$= \{x \mid$ Es gibt eine Zahl $k \in \mathbb{N} : 2k = x\}$\\
        
        \item $L_2 = \{a^n \mid n \in \mathbb{N}\} = \{\emptyWord, a,aa,aaa,aaaa, \dots\}$
        
        \item $L_3 = \{a^nb^n \mid n \in \mathbb{N}\} = \{\emptyWord, ab ,aabb,aaabbb,aaaabbbb, \dots\}$
        
        \item $L_4 = \{a^nw \mid n \in \mathbb{N}, w = bccb\} = \{bccb, abccb, aabccb, \dots\}$\\
        $L_4$ endet nach einer beliebigen Anzahl von a's immer mit bccb
        
        \item $L_5 = \{w \mid |w| = 2, w\in \{a,b\}^{*}\} = \{aa,bb,ab,ba\}$\\
        Wörter der Länge 2 aus $\{a,b\}^{*}$
        
        \item $L_6 = \{w \mid |w|_a = 2, w\in \{a,b\}^{*}\}$\\
        Wörter mit \alert{genau} 2 a's aus $\{a,b\}^{*}$
    \end{itemize}
\end{frame}

{\setbeamercolor{palette primary}{bg=ExColor}
\begin{frame}[fragile]{Denkpause}
    \footnotesize
        \begin{alertblock}{Aufgaben}
            Findet Wörter aus den folgenden Sprachen
        \end{alertblock}
        \metroset{block=fill}
        \begin{block}{Normal}
            \begin{itemize}
                \item $L_1 = \{a\}$
                \item $L_2 = \{uv\;|\;u\in\{a,b\}^\ast,\;v\in\{c,d\}\}$
                \item $L_3 = \{w\;|\;|w| = 3, w\in \{a,b,c\}^{*}\}$
            \end{itemize}
        \end{block}
        \begin{block}{Etwas Schwerer}
            \begin{itemize}
                \item $L_4 = \{a^n\;|\;n \equiv 1 \bmod 3, n\in\mathbb{N}\}$
                \item $L_5 = \{w\;|\;|w|_a = 3, |w|_b = 1, w\in \{a,b,c\}^{*}\}$
                \item $L_6 = \{uv\;|\;u\in\{\text{\Rewind, \MoveUp, \Forward, \MoveDown}\}^\ast,\;v\in\{\text{\Stopsign}\}\}$
                \item $L_7 = \{w \mid |w| = 2, w\in \{a,b\}\}$
            \end{itemize}
        \end{block}
        \emph{Anmerkung:} $z \equiv x \bmod y \iff z = n \cdot y + x$, mit $n,x,y,z \in \mathbb{Z}$
\end{frame}
}

{\setbeamercolor{palette primary}{bg=ExColor}
\begin{frame}{Lösungen}
  \begin{itemize}[<+- | alert@+>]
        \item 
            $L_1$: Enthält \textbf{nur} das einzelne Wort a!
        \item
            $L_2$: z.B. c, d, ac, bc, aaac, abababad, \dots\\
            Wort besteht aus zwei Teilen: u z.B. $\emptyWord$, a, b, ababa, ...\\ v ist entweder c oder d!
        \item
            $L_3$: enthält alle Wörter der Länge 3, deren Buchstaben nur a, b oder c sind.\\
            $\rightarrow$ aaa, aab, aba, abb, abc, acb, acc, baa, bab, cbb, ...
        \item
            $L_4 = \{a, aaaa, aaaaaaa, \dots\}$\\
            Wörter deren Länge durch 3 geteilt den Rest 1 ergeben.
        \item
            $L_5 = \{caaba, cccbaaa, abaca, aaab, \dots\}$\\
            genau 3 a's, genau 1 b, beliebig viele c's, keine Sortierung
        \item
            $L_6$ = \{\Stopsign, \Rewind \Stopsign, \MoveUp \Stopsign,\dots\;, \MoveDown \Rewind \MoveDown \Stopsign,\dots\}
            \item $L_7$: Enthält \textbf{gar kein} Wort!
    \end{itemize}
\end{frame}
}
