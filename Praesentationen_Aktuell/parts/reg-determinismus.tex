\subsection{DEA}
\begin{frame}[fragile]{Deterministische endliche Automaten}
\begin{alertblock}{Wir beschränken unseren Automaten folgendermaßen:}
Von \alert{jedem Zustand} muss \alert{genau ein} Übergang für \alert{jedes $a\in\Sigma$} ausgehen.
\end{alertblock}
%\vspace{0.3}
\onslide<2->{
    Um dies zu ermöglichen führen wir eine neue Komponente ein:
    \begin{alertblock}{Fangzustand}
        Dieser Zustand kann nicht verlassen werden.\\
        Falls der Automat in diesem Zustand landet, kommt er nicht mehr raus. Das Wort kann nicht akzeptiert werden.
        \begin{center}
        \begin{tikzpicture}[->,>=stealth',shorten >=1pt,auto,node distance=2cm,semithick]
        \node[state](qi){$\emptyset$};
        \path (qi) edge [loop right] node {$x \in \Sigma$} (B);
        \end{tikzpicture}
        \end{center}
    \end{alertblock}
}
\end{frame}

\begin{frame}[fragile]{Beispiel}
\begin{exampleblock}{$L=\{axb \mid x\in\{a,b\}^\ast\}$}
\vspace{0.3cm}
        \begin{tikzpicture}[->,>=stealth,shorten >=1pt,auto,node distance=2cm,
                        semithick]
        %\tikzstyle{every state}=[fill=ExColor,draw=none,text=white]
        
        \node<1,3-> [initial,state]         (1)               {$q_0$};
        \node<2>    [initial,state,orange]  (1)               {$q_0$};
        \node       [state]                 (2) [right of=1]  {$q_1$};
        \node<1-8>  [state,accepting]       (3) [right of=2]  {$q_E$};
        \node<9>    [state,accepting,orange](3) [right of=2]  {$q_E$};
        \node       [state]                 (4) [below of=1]  {$\emptyset$};
        
        \path<1,2,9>
                (1) edge                node {a}  (2)
                    edge                node {b}  (4)
                (2) edge [loop above]   node {a}  (2)
                    edge [bend left]    node {b}  (3)
                (4) edge [loop right]   node {a,b}(4)
                (3) edge [loop above]   node {b}  (3)
                    edge [bend left]    node {a}  (2);
        \path<3>
                (1) edge [orange]       node {a}  (2)
                    edge                node {b}  (4)
                (2) edge [loop above]   node {a}  (2)
                    edge [bend left]    node {b}  (3)
                (4) edge [loop right]   node {a,b}(4)
                (3) edge [loop above]   node {b}  (3)
                    edge [bend left]    node {a}  (2);
        \path<4>
                (1) edge                node {a}  (2)
                    edge                node {b}  (4)
                (2) edge [loop above,orange] node {a}  (2)
                    edge [bend left]    node {b}  (3)
                (4) edge [loop right]   node {a,b}(4)
                (3) edge [loop above]   node {b}  (3)
                    edge [bend left]    node {a}  (2);
        \path<5,7>
                (1) edge                node {a}  (2)
                    edge                node {b}  (4)
                (2) edge [loop above]   node {a}  (2)
                    edge [bend left,orange] node {b}  (3)
                (4) edge [loop right]   node {a,b}(4)
                (3) edge [loop above]   node {b}  (3)
                    edge [bend left]    node {a}  (2);
        \path<6>
                (1) edge                node {a}  (2)
                    edge                node {b}  (4)
                (2) edge [loop above]   node {a}  (2)
                    edge [bend left]    node {b}  (3)
                (4) edge [loop right]   node {a,b}(4)
                (3) edge [loop above]   node {b}  (3)
                    edge [bend left,orange] node {a}  (2);
        \path<8>
                (1) edge                node {a}  (2)
                    edge                node {b}  (4)
                (2) edge [loop above]   node {a}  (2)
                    edge [bend left]    node {b}  (3)
                (4) edge [loop right]   node {a,b}(4)
                (3) edge [loop above,orange] node {b}  (3)
                    edge [bend left]    node {a}  (2);
        \end{tikzpicture}
\end{exampleblock}
\onslide<2->{
    \begin{exampleblock}{Worteingabe:} \alert<3>{a}\alert<4>{a}\alert<5>{b}\alert<6>{a}\alert<7>{b}\alert<8>{b} $\in L$\only<1-8>{?} \only<9>{$\leadsto$ \alert{akzeptiert}
}
\end{exampleblock}}
\end{frame}

{\setbeamercolor{palette primary}{bg=ExColor}
\begin{frame}{Denkpause}
\footnotesize
    \begin{alertblock}{Aufgaben}
    Finde deterministische endliche Automaten (DEAs) für die folgenden Sprachen.
    \end{alertblock}
    \metroset{block=fill}
    \begin{block}{Normal}
    \begin{itemize}
        \item $L_1 = \{a^{2n} \mid n \in \mathbb{N}\}$ über $\Sigma=\{a\}$
        \item $L_2 = \{w\in \{a,b\}^* \mid |w|_a \geq 2\}$ über $\Sigma=\{a,b\}$
    \end{itemize}
    \end{block}
    \begin{block}{Prüfungsaufgabe: Etwas Schwerer}
    \begin{itemize}
        \item $L_3 = \{w \in \{a, b, c\}^* \mid \; |w|_b \geq 1$ und $aaca$ ist Suffix von w$\}$ über $\Sigma=\{a,b,c\}$
    \end{itemize}
    \end{block}
    \begin{block}{Prüfungsaufgabe: Schwer}
    \begin{itemize}
        \item $L_4 = \{bin(n)\mid \exists k \in \mathbb{N}: n=4k, bin(n)\text{ ist Binärdarstellung von }n\}$ über $\Sigma=\{1,0\}$\\
        \emph{Achtung: Keine führenden Nullen.}
    \end{itemize}
    \end{block}
\end{frame}
}

{\setbeamercolor{palette primary}{bg=ExColor}
\begin{frame}{Lösung: Normal}
    \only<1->{
    \begin{alertblock}{$L_1 = \{a^{2n} \mid n \in \mathbb{N}\}$}
        \begin{tikzpicture}[->,>=stealth',shorten >=1pt,auto,node distance=2cm,
                            semithick]
        \node [initial,state,accepting] (0)              {$q_0$};
        \node [state]                   (1) [right of=0] {$q_1$};
        \path   (0) edge [bend left]   node {a} (1)
                (1) edge [bend left]   node {a} (0);
        \end{tikzpicture}
    \end{alertblock}}
    
    \only<2->{
    \begin{alertblock}{$L_2 = \{w\in \{a,b\}^* \mid |w|_a \geq 2\}$}
        \begin{tikzpicture}[->,>=stealth',shorten >=1pt,auto,node distance=2cm,
                            semithick]
        \node [initial,state]   (0)              {$q_0$};
        \node [state]           (1) [right of=0] {$q_1$};
        \node [state,accepting] (2) [right of=1] {$q_2$};
        \path   (0) edge               node {a} (1)
                    edge [loop above]  node {b} (1)
                (1) edge               node {a} (2)
                    edge [loop above]  node {b} (2)
                (2) edge [loop above]  node {a,b}(2);
        \end{tikzpicture}
    \end{alertblock}}
\end{frame}
}

{\setbeamercolor{palette primary}{bg=ExColor}
\begin{frame}{Lösung: Prüfungsaufgabe: Etwas Schwerer}
    \begin{alertblock}{$L_3 = \{w \in \{a, b, c\}^* \mid \; |w|_b \geq 1$ und $aaca$ ist Suffix von w$\}$}
        \begin{tikzpicture}[->,>=stealth',shorten >=1pt,auto,node distance=1.5cm,
                            semithick]
        \node [initial,state]   (0)              {$q_0$};
        \node [state]           (1) [right of=0] {$q_1$};
        \node [state]           (2) [right of=1] {$q_2$};
        \node [state]           (3) [right of=2] {$q_3$};
        \node [state]           (4) [right of=3] {$q_4$};
        \node [state,accepting] (5) [right of=4] {$q_E$};
        
        \path
                (0) edge                    node {b}    (1)
                    edge [loop below]       node {a,c}  (0)
                (1) edge                    node {a}    (2)
                    edge [loop below]       node {b,c}  (1)
                (2) edge                    node {a}    (3)
                    edge [bend right=85]    node {b,c}  (1)
                (3) edge                    node {c}    (4)
                    edge [bend left]        node {b}    (1)
                    edge [loop below]       node {a}    (3)
                (4) edge                    node {a}    (5)
                    edge [bend right=95]    node {b,c}  (1)
                (5) edge [bend left=60]     node {b,c}  (1)
                    edge [bend left]        node {a}    (3);
        \end{tikzpicture}
    \end{alertblock}
\end{frame}

\begin{frame}{Lösung: Prüfungsaufgabe: Schwer}
    \begin{alertblock}{$L_4 = \{bin(n)\mid \exists k \in \mathbb{N}: n=4k, bin(n)\text{ ist Binärdarstellung von }n\}$}
        \begin{tikzpicture}[->,>=stealth',shorten >=1pt,auto,node distance=2cm,
                            semithick]
        \node [initial,state]   (0)              {$q_0$};
        \node [state]           (1) [right of=0] {$q_1$};
        \node [state]           (2) [right of=1] {$q_2$};
        \node [state,accepting] (3) [below of=2] {$q_3$};
        \node [state,accepting] (4) [below of=0] {$q_4$};
        \node [state]           (5) [below of=1] {$F$};
        \path   (0) edge               node {1} (1)
                    edge               node {0} (4)
                (1) edge               node {0} (2)
                    edge [loop above]  node {1} (1)
                (2) edge               node {0} (3)
                    edge [bend left]  node {1} (1)
                (3) edge [loop right]  node {0} (3)
                    edge               node {1} (1)
                (4) edge               node {0,1}(5)
                (5) edge [loop below]  node {0,1}(5);
        \end{tikzpicture}
    \end{alertblock}
\end{frame}
}

% {\setbeamercolor{palette primary}{bg=ExColor}
% \begin{frame}{Denkpause}
%     \begin{alertblock}{Aufgaben}
%     Finde einen deterministischen Automaten für die folgenden Sprachen:
%     \end{alertblock}
%     \metroset{block=fill}
%     \begin{block}{Normal}
%     \begin{itemize}
%         \item $L_1 = \{a^{2n} \mid n\in\naturals\}$
%         \item $L_2 = \{a^nb^m \mid n, m\in\naturals\}$
%         \item $L_3 = \{uv \mid u\in\{a,b\}^\ast,\;v\in\{c,d\}\}$
%         \item $L_4 = \{w \mid |w| = 3, w\in \{a,b,c\}^*\}$
%     \end{itemize}
%     \end{block}
%     \begin{block}{Etwas Schwerer}
%     \begin{itemize}
%         \item $L_4 = \{a^n \mid n \equiv 1 \mod 3\}$
%         \item $L_5 = \{w \mid |w|_a = 3, |w|_b = 1, w\in \{a,b,c\}^*\}$
%         \item $L_6 = \{uv \mid u\in\{\text{\Rewind, \MoveUp, \Forward, \MoveDown}\}^\ast,\;v\in\{\text{\Stopsign}\}\}$
%         \item 
%     \end{itemize}
%     \end{block}
% \end{frame}
% }

% {\setbeamercolor{palette primary}{bg=ExColor}
% \begin{frame}{Lösungen}
% Alle Lösungen sind Beispiellösungen, es sind auch andere möglich.
%     \begin{itemize}[<+- | alert@+>]
%         \item bla
%     \end{itemize}
% \end{frame}
% }  



%ToDo
% - gute Aufgaben zu Automaten finden, die Spaß machen zu bearbeiten und interesse an theo fördern
%       - zB richtige Klammerung, Aufzug etc.
%       - grammatik labyrinth aufgabe nochmal, nur mit automaten
% - erst nea, dann dea, minimal nicht erwähnen
% - NEA vs DEA?
% - typ 3 grammatik kann das selbe wie endlicher automat
% - je nach zeit noch reguläre ausdrücke
% - möglichst so konstruieren, dass man an einige punkte hat an denen man bedenkenlos aus Zeitgründen aufhören kann
