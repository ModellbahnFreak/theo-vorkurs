\begin{frame}{Weiterer Mengenbeweis}
    Ein weiterer Mengenbeweis...
    \metroset{block=fill}
    \begin{block}{\alert{Aufgabe}}
    $L_1=\{w^{n} \mid n \in \mathbb{N}, w \in \{aaaa\}\}$\\
    $L_2=\{w \mid |w| \equiv 0 \bmod 4, w \in \{a\}^*\}$
    \end{block}
    Zu zeigen:\\
    $L_1 = L_2$\\
    d.h. $(\forall x: x \in L_1 \implies x \in L_2) \wedge (\forall x: x \in L_2 \implies x \in L_1)$

\end{frame}

\begin{frame}{Weiterer Mengenbeweis}
    \metroset{block=fill}
    \only<1>{
    \begin{block}{\alert{$\forall x: x \in L_1 \implies x \in L_2$}}
    Sei $x$ beliebig.\\
    Angenommen, $x \in L_1$.\\
    Es gilt: $w \in \{aaaa\}$. Damit gilt $|w|=4$. Es folgt $|x| = |w^{n}| = |w| * n = 4 * n$ mit $n \in \mathbb{N}$. Daraus folgt $|x| \equiv 0 \bmod 4$. Weiterhin gilt $(aaaa)^n \in \{a\}^*$.\\
    $\leadsto x \in L_2$
    \end{block}
    }
    \only<2>{
    \begin{block}{\alert{$\forall x: x \in L_2 \implies x \in L_1$}}
    Sei $x$ beliebig.\\
    Angenommen, $x \in L_2$.\\
    Es gilt: $|x| \equiv 0 \bmod 4$. \\ Damit gilt $|x| = 4 * n = |w| * n = |w^n|$ mit $w \in\{aaaa\}$ und $n \in \mathbb{N}$. \\ Weiterhin gilt $w \in \{a\}^*$. \\
    $\leadsto x \in L_1$
    \end{block}
    }
    \only<3>{
    Da gezeigt wurde:\\
    \vspace{0.3cm}
    $\forall x: x \in L_1 \implies x \in L_2$\\
    \alert{und}\\
    $\forall x: x \in L_2 \implies x \in L_1$\\
    \vspace{0.3cm}
    \textbf{gilt $L_1 = L_2$.}
    }
\end{frame}

{\setbeamercolor{palette primary}{bg=ExColor}
\begin{frame}[fragile]{Denkpause}
    \begin{alertblock}{Aufgaben}
    Versuche dich an folgenden Mengenbeweisen.
    \end{alertblock}
    
    \metroset{block=fill}
    \begin{block}{Etwas Schwerer}
    $L_1=\{a^{n}b^{m} \mid n<m ,mit \; n,m\in \mathbb{N}\}$\\
    $L_2=\{w \mid |w|_a < |w|_b, w \in \{a,b\}^*\}$\\
    \vspace{0.3cm}
    Zu zeigen: $L_1 \subsetneq L_2$
    \end{block}
    \metroset{block=fill}
    \begin{block}{Schwer}
        $L_1 = \{a^nb^n \mid n \in \mathbb{N}\}$\\
        $L_2 = \{w \in \{a, b\}^* \mid |w|_a = |w|_b\}$\\
        $L_3 = \{a^kb^l \mid k,l \in \mathbb{N}\}$\\
        \vspace{0.3cm}
        Zu zeigen: $L_1 = L_2 \cap L_3$
    \end{block}
\end{frame}
}

{\setbeamercolor{palette primary}{bg=ExColor}
\begin{frame}[fragile]{Lösung}
    \begin{alertblock}{Aufgaben}
    z.Z. $L_1 \subsetneq L_2$\\
    d.h. $(\forall x: x \in L_1 \implies x \in L_2) \wedge (L_1 \neq L_2)$
    \end{alertblock}
    
    \begin{columns}
    \column{0.6\textwidth}
    \metroset{block=fill}
    \begin{alertblock}{$\forall x: x \in L_1 \implies x \in L_2$}
        Sei x beliebig. Ang. $x \in L_1$.\\
        Es gilt: $x=a^{n}b^{m}$, mit $n,m\in\mathbb{N}$.\\
        Damit gilt $|x|_a=n, |x|_b=m$ mit $n<m$.\\
        Also auch $|x|_a < |x|_b$.\\
        $\leadsto x \in L_2$.
    \end{alertblock}
    
    \column{0.4\textwidth}
    \metroset{block=fill}
    \begin{alertblock}{$L_1 \neq L_2$}
        Beweis durch Angabe eines Gegenbeispiels:\\
        $bba \in L_2$, aber $bba \notin L_1$\\
        Also sind $L_1$ und $L_2$ nicht gleich.
    \end{alertblock}
    \end{columns}
    \qed
\end{frame}
}

{\setbeamercolor{palette primary}{bg=ExColor}
\begin{frame}[fragile]{Lösung}
    \begin{alertblock}{Aufgaben}
        Z.z. $L_1 = L_2 \cap L_3$\\
        d.h. $\forall w \in \{a,b\}^*: w \in L_1 \iff w \in L_2 \wedge w \in L_3$
    \end{alertblock}
    \only<1-4>{
        \metroset{block=fill}
        \begin{alertblock}{\glqq$\implies$\grqq}
            \begin{enumerate}
                \item<1-4> Sei $w \in L_1$ beliebig.
                \item<2-4> Dann existiert ein $n \in \mathbb{N}$ so, dass $w = a^nb^n$
                \item<3-4> Insbesondere gilt:
                \begin{enumerate}
                    \item[i)] $|w|_a = |w|_b$
                    \item[ii)] $w = a^kb^l$ mit $k := n =: l$
                \end{enumerate}
                \item<4> Folglich ist also $w \in L_2$ und $w \in L_3$.
            \end{enumerate}
        \end{alertblock}
    }
    \only<5->{
        \metroset{block=fill}
        \begin{alertblock}{\glqq$\impliedby$\grqq}
            \begin{enumerate}
                \item<5-> Sei $w \in L_2 \cap L_3$ beliebig.
                \item<6->  Dann gilt:
                \begin{enumerate}
                    \item[i)] $n := |w|_a = |w|_b$
                    \item[ii)] $w = a^kb^l$ für $k, l \in \mathbb{N}$
                \end{enumerate}
                \item<7->  Und somit $w = a^{|w|_a}b^{|w|_b} = a^nb^n$.
                \item<8->  Folglich ist $w \in L_1$
            \end{enumerate}
        \end{alertblock}
    }
\end{frame}
}
