\documentclass[aspectratio=43,10pt]{beamer}

\usetheme[progressbar=frametitle]{metropolis}
\usepackage{appendixnumberbeamer}
\usepackage[ngerman]{babel}
\usepackage[utf8]{inputenc}
\usepackage{t1enc}

\usepackage{booktabs}
\usepackage[scale=2]{ccicons}
\usepackage{hyperref}

\usepackage{pgf}
\usepackage{tikz}
\usetikzlibrary{arrows,automata,positioning}
\usepackage{pgfplots}
\usepgfplotslibrary{dateplot}

\usepackage{xspace}
\newcommand{\themename}{\textbf{\textsc{metropolis}}\xspace}

\usepackage{blindtext}
\usepackage{graphicx}
\usepackage{subcaption}
\usepackage{comment}
\usepackage{mathtools}
\usepackage{amsmath}
\usepackage{centernot}
\usepackage{amssymb}
\usepackage{proof}
\usepackage{tabularx}
\renewcommand{\figurename}{Abb.}
\usepackage{marvosym}
\usepackage{mathtools}
\usepackage{qrcode}

\definecolor{ExColor}{HTML}{17819b}

\newcommand{\emptyWord}{\varepsilon}
\newcommand{\SigmaStern}{\Sigma^{*}}
\newcommand{\absval}[1]{|#1|}
\newcommand{\defeq}{\vcentcolon=}
\newcommand{\eqdef}{=\vcentcolon}
\newcommand{\nimplies}{\centernot\implies}

\newcommand{\naturals}{\ensuremath{\mathbb{N}}}
\newcommand{\integers}{\ensuremath{\mathbb{Z}}}
\newcommand{\rationals}{\ensuremath{\mathbb{Q}}}
\newcommand{\reals}{\ensuremath{\mathbb{R}}}

\setbeamertemplate{footline}[text line]
{\parbox{\linewidth}{Fachgruppe Informatik\hfill\insertpagenumber\hfill Vorkurs Theoretische Informatik\vspace{0.2in}}}


\title{Vorkurs Theoretische Informatik}
\subtitle{Einführung in reguläre Sprachen}
\date{Freitag, 16.10.2020}
\author{Arbeitskreis  Theo Vorkurs}
\institute{Fachgruppe Informatik}
% \titlegraphic{\hfill\includegraphics[height=1.5cm]{logo.pdf}}

\begin{document}

\maketitle

\begin{frame}[fragile]{Übersicht}
  \setbeamertemplate{section in toc}[sections numbered]
  \tableofcontents%[hideallsubsections]
\end{frame}


\section{Grammatik und Automaten}

\begin{frame}{Automaten: Formal}
    Ein \textbf{DEA M} lässt sich beschreiben durch ein geordnetes 5-Tupel\\
    \alert{$M=(Z, \Sigma, \delta, z_0, E)$} mit:
    \begin{itemize}
        \item $Z$: Die Menge der Zustände
        \item $\Sigma$: Das Alphabet
        \item $\delta$: Die Überführungsfunktion
        \item $z_0$: Der Startzustand
        \item $E$: Die Menge der Endzustände
    \end{itemize}
\end{frame}


\begin{frame}{Automaten: Formal}
\begin{columns}
    \column{0.5\textwidth}
        \alert{$L_1=\{a^{n}b^{m} \mid n,m \in \mathbb{N}\}$}\\
        \vspace{0.6cm}
        \begin{tikzpicture}[->,>=stealth',shorten >=1pt,auto,node distance=1.5cm,semithick]
        \node [initial,state,accepting]   (0)              {$q_0$};
        \node [state,accepting]           (1) [right of=0] {$q_1$};
        \node [state]                     (2) [below of=0] {$F$};
        
        \path   (0) edge                    node {b}    (1)
                    edge [loop above]       node {a}    (0)
                (1) edge                    node {a}    (2)
                    edge [loop above]       node {b}    (1)
                (2) edge [loop right]       node {a,b}  (2);
        \end{tikzpicture}

    \column{0.5\textwidth}
    \alert{$M=(Z, \Sigma, \delta, q_0, E)$} mit:
    \begin{itemize}
        \item<2-> \alert<2>{$Z=\{q_0, q_1, F\}$}
        \item<3-> \alert<3>{$\Sigma=\{a, b\}$}
        \item<4-> \alert<4>{$\delta$: 
        \begin{itemize}
            \item $\delta(q_0, a)=q_0$
            \item $\delta(q_0, b)=q_1$
            \item $\delta(q_1, a)=F$
            \item $\delta(q_1, b)=q_1$
            \item $\delta(F, a)=F$
            \item $\delta(F, b)=F$
        \end{itemize}}
        \item<5-> \alert<5>{$E=\{q_0, q_1\}$}
    \end{itemize}
\end{columns}
\end{frame}

\begin{frame}{Kurz selbst denken...}
\begin{columns}
    \column{0.5\textwidth}
        \alert{$L_2=\{w \in \{a, b\}^* \mid |w|=3\}$}\\
        %TÖDÖ Töröööööt Benjamin der kleiner Elefant fragt Janette: Ey machst du automat amk!
        
        \vspace{0.6cm}
        \begin{tikzpicture}[->,>=stealth',shorten >=1pt,auto,node distance=1.5cm,semithick]
        \node [initial,state]   (0)              {$q_0$};
        \node [state]           (1) [below of=0] {$q_1$};
        \node [state]           (2) [below of=1] {$q_2$};
        \node [state,accepting] (3) [below of=2] {$q_E$};
        \node [state]           (f) [right of=3] {$F$};
        
        \path   (0) edge                node {a,b}  (1)
                (1) edge                node {a,b}  (2)
                (2) edge                node {a,b}  (3)
                (3) edge                node {a,b}  (f)
                (f) edge [loop above]   node {a,b} (f);
        \end{tikzpicture}

    \column{0.5\textwidth}
    \alert{$M=(Z, \Sigma, \delta, q_0, E)$} mit:
    \begin{itemize}
        \item<2-> \alert<2>{$Z=\{q_0, q_1, q_2, q_E, F\}$}
        \item<3-> \alert<3>{$\Sigma=\{a, b\}$}
        \item<4-> \alert<4>{$\delta$: 
        \begin{itemize}
            \item $\delta(q_0, a)=q_1$
            \item $\delta(q_0, b)=q_1$
            \item $\delta(q_1, a)=q_2$
            \item $\delta(q_1, b)=q_2$
            \item $\delta(q_2, a)=q_3$
            \item $\delta(q_2, b)=q_3$
            \item $\delta(q_E, a)=F$
            \item $\delta(q_E, b)=F$
            \item $\delta(F, a)=F$
            \item $\delta(F, b)=F$
        \end{itemize}}
        \item<5-> \alert<5>{$E=\{q_E\}$}
    \end{itemize}
\end{columns}
\end{frame}

\begin{frame}[fragile]{Automaten und Grammatiken}
    \metroset{block=fill}
    \begin{exampleblock}{Satz}
    Jede durch endliche Automaten erkennbare Sprache ist auch regulär (also Typ 3).
    \end{exampleblock}
   Sei \alert{$A \subseteq \SigmaStern$} eine Sprache und \alert{M ein Automat mit \textbf{T(M) = A}},\\ \emph{(d.h. M erkennt die Sprache A)}.\\
   \vspace{0.3cm}
   Wir definieren eine \alert{Typ 3-Grammatik G mit \textbf{L(G)=A}},\\ \emph{(d.h. die Grammatik G erzeugt die Sprache A)}.\\
   \vspace{0.3cm}
   Es ist $G=(V, \Sigma, P, S)$ mit:\\
   V = Menge der Zustände des Automaten $(Z)$\\
   S = Startzustand des Automaten $(z_0)$\\
    \vspace{0.3cm}
   \only<1>{\emph{Falls $\emptyWord \in A$, dann enthält P die Regel \glqq$z_0 \to \emptyWord$\grqq}}
\end{frame}

\begin{frame}[fragile]{Automaten und Grammatiken}
    \begin{columns}
        \column{0.5\textwidth}
        Unsere Menge der Produktionsregeln P besteht aus folgenden Regeln:\\
        \vspace{0.3cm}
        Jeder \glqq$\delta$-Anweisung\grqq\ \alert{$\delta(z_1, a)=z_2$} ordnen wir folgende Regeln zu.
        \begin{itemize}
            \item \alert<2>{$z_1 \rightarrow a z_2$}
            \item Und zusätzlich, falls \alert<3>{$z_2 \in E: z_1 \rightarrow a$}
        \end{itemize}
        \column{0.5\textwidth}
        \centering
            \only<2>{\alert<2>{            \begin{tikzpicture}[->,>=stealth',shorten >=1pt,auto,node distance=2cm,semithick]
            \node [state] (1)  {$z_1$};
            \node [state] (2) [right of=1]  {$z_2$};
            \path (1) edge node {a} (2);
            \end{tikzpicture}\\
            \glqq$\delta (z_1,a) = z_2$\grqq}}
            \only<3>{\alert<3>{
            \begin{tikzpicture}[->,>=stealth',shorten >=1pt,auto,node distance=2cm,semithick]
            \node [state] (1)  {$z_1$};
            \node [state, accepting] (2) [right of=1]  {$z_2$};
            \path (1) edge node {a} (2);
            \end{tikzpicture}\\
            \glqq$\delta (z_1,a) = z_2$\grqq}}
            
    \end{columns}
\end{frame}

\begin{frame}[fragile]{Automaten und Grammatiken}
    \begin{alertblock}{Zu zeigen: $x \in T(M)$ gdw. $x \in L(G)$}
    Dabei gilt: $x=a_1 a_2 ... a_n$\\
    Die folgenden Aussagen sind äquivalent:
        \begin{itemize}
            \item x wird von Automat M erkannt \emph{$(x\in T(M))$}
            \item Es gibt eine Folge von Zuständen $z_0, z_1, \dots, z_n$ mit: $z_0$ ist Startzustand, $z_n$ ist Endzustand \textbf{und}: $\forall i \in \{1, ..., n\}: \delta(z_{i-1}, a_i)=z_i$
            \item Es gibt Folge an Variablen $z_0, z_1, \dots, z_n$ mit: $z_0$ ist Startvariable und x lässt sich von $z_0$ ausgehend ableiten.
            \item x wird von der Grammatik G produziert \emph{$(x \in L(G))$}
        \end{itemize}
        \qed
    \end{alertblock}
\end{frame}

% \begin{frame}[fragile]{Automaten und Grammatiken}
%     \begin{columns}
%         \column{0.5\textwidth}
%         P besteht aus folgenden Regeln:\\
%         \vspace{0.3cm}
%         Jeder \glqq$\delta$-Anweisung\grqq\ \alert{$\delta(z_1, a)=z_2$} ordnen wir folgende Regeln zu.
%         \begin{itemize}
%             \item \alert<2>{$z_1 \rightarrow a z_2$}
%             \item Und zusätzlich, falls \alert<3>{$z_2 \in E: z_1 \rightarrow a$}
%         \end{itemize}
%         \column{0.5\textwidth}
%     \end{columns}
% \end{frame}


% \begin{frame}[fragile]{Grammatik zu Automaten}
%     Eine reguläre Grammatik und eine endlicher Automat können die selben Sprachen beschreiben.
%     \begin{exampleblock}{\glqq$\implies$\grqq}
%     \only<1>{Aus $z_1 \to az_2 \in P$ wird:\\
%     \begin{center}
%     \vspace{0.3cm}
%         \begin{tikzpicture}[->,>=stealth',shorten >=1pt,auto,node distance=2cm,
%                         semithick]
%         %\tikzstyle{every state}=[fill=ExColor,draw=none,text=white]
        
%         \node       [state]                 (1) [right of=1]  {$z_1$};
%         \node       [state]                 (2) [right of=2]  {$z_2$};
        
%         \path
%                 (1) edge                node {a}  (2);
%         \end{tikzpicture}\\
%         \glqq$\delta (z_1,a) = z_2$\grqq
%     \end{center}}
    
%     \only<2>{
%     Falls $z_2 \in F$ ($z_2$ ein Endzustand)\\
%     Aus $z_1 \to a \in P$ wird:\\
%     \begin{center}
%     \vspace{0.3cm}
%         \begin{tikzpicture}[->,>=stealth',shorten >=1pt,auto,node distance=2cm,
%                         semithick]
%         %\tikzstyle{every state}=[fill=ExColor,draw=none,text=white]
        
%         \node       [state]                 (1) [right of=1]  {$z_1$};
%         \node       [state, accepting]      (2) [right of=2]  {$z_2$};
        
%         \path
%                 (1) edge                node {a}  (2);
%         \end{tikzpicture}\\
%         \glqq$\delta (z_1,a) = z_2$\grqq
%     \end{center}
%     }
%     \end{exampleblock}
% \end{frame}


% \begin{frame}[fragile]{Automaten zu Grammatik}
%   Eine reguläre Grammatik und eine endlicher Automat können die selben Sprachen beschreiben.
%   \begin{exampleblock}{\glqq$\impliedby$\grqq}
%   bla
%   \end{exampleblock}
% \end{frame}


\begin{frame}[standout]
  Verdauungspause
\end{frame}

\section{Reguläre Ausdrücke}

\begin{frame}[fragile]{mehr Möglichkeiten reguläre Sprachen zu beschreiben}
Graphen und \glqq Bilder\grqq\ sind oft nicht das optimale Mittel eine Sprache zu beschreiben.\\
Die \alert{regulären Ausdrücke} bieten uns eine Möglichkeit Sprachen schnell und intuitiv zu beschreiben.
\begin{alertblock}{Funktionsweise}
\begin{enumerate}
    \item Wörter können mit einem angegebenen Muster abgeglichen werden.
    \item Lässt sich ein Wort durch das Muster beschreiben, ist es in der davon beschriebenen Sprache. 
\end{enumerate}
\end{alertblock}
\end{frame}

\begin{frame}{Reguläre Ausdrücke verwenden}
    \begin{alertblock}{Induktive Definition der Syntax}
    \begin{itemize}
       \item \alert{$\emptyset$} und \alert{$\emptyWord$} sind reguläre Ausdrücke.
       \item \alert{a} ist ein regulärer Ausdruck (für alle a $\in \Sigma$).
       \item Wenn \alert{$\alpha$} und \alert{$\beta$} reguläre Ausdrücke sind,\\dann sind \alert{$\alpha \beta$}, \alert{($\alpha \mid \beta$)} und \alert{$(\alpha)^*$} auch reguläre Ausdrücke.
   \end{itemize}
   \end{alertblock}
   \metroset{block=fill}
   \begin{exampleblock}{Beispiel}
   $\gamma = ((a|b)^* \mid \emptyWord) \implies aba \in L(\gamma)$
   \end{exampleblock}
\end{frame}

\begin{frame}{Reguläre Ausdrücke verwenden}
    \begin{exampleblock}{Wie Sprachen und reguläre Ausdrücke zusammenhängen}
    \footnotesize
    \begin{itemize}[<+- | alert@+>]
        \item Wenn $\gamma = \emptyset$, beschreibt es die leere Sprache: $L(\gamma) = \{\}$
        \item Wenn $\gamma$ ein einzelnes Wort ist, ist genau dieses Wort in der Sprache enthalten.\\
        $\gamma = \emptyWord$: $L(\gamma)=\{\emptyWord\}$,\quad$\gamma = a$: $L(\gamma)=\{a\}$
        \item Wenn $\gamma$ aus zwei hintereinandergeschrieben Ausdrücken besteht, repräsentiert Konkatenation.\\
        $\gamma = (a)^*(b)^*: L(\gamma)=\{a^nb^m|n,m\in\mathbb{N}\}$
        \item Wenn $\gamma$ aus zwei mit \glqq oder\grqq\ verknüpften Ausdrücken besteht, sind beide Seiten in der Sprache enthalten.\\
        $\gamma = (a \mid bc)$: $L(\gamma)=\{a, bc\}$
        \item Wenn $\gamma$ ein Ausdruck mit einem Stern ist, kann dieser innere Ausdruck beliebig oft wiederholt werden (auch null mal).\\
        $\gamma = (a)^*$: $L(\gamma)=\{\emptyWord, a, aa, aaa, ...\}=\{a\}^*$
        \item Alles zusammen --- $\gamma = ((a)^* \mid (bc)^*)$: $L(\gamma)=\{\emptyWord, a, bc, aa, bcbc, aaa, ...\} = \{a\}^* \cup \{bc\}^*$
    \end{itemize}
    \end{exampleblock}
\end{frame}

{\setbeamercolor{palette primary}{bg=ExColor}
\begin{frame}{Reguläre Ausdrücke}
    \begin{alertblock}{Aufgaben}
    Finde einen regulären Ausdruck für die folgenden Sprachen
    \end{alertblock}
    \metroset{block=fill}
    \begin{block}{Normal}
    \begin{itemize}
        \item $L(\gamma_1) = \{a^{2n} \mid n\in\naturals\}$
        \item $L(\gamma_2) = \{a^nb^m \mid n, m\in\naturals\}$
        \item $L(\gamma_3) = \{uv \mid u\in\{a,b\}^\ast,\ v\in\{c,d\}\}$
        \item $L(\gamma_4) = \{w \mid |w| = 3, w\in \{a,b,c\}^*\}$
    \end{itemize}
    \end{block}
    \begin{block}{Etwas Schwerer}
    \begin{itemize}
        \item $L(\gamma_5) = \{a^n \mid n \equiv 1 \mod 3\}$
        \item $L(\gamma_6) = \{uv\mid u\in\{\text{\Rewind, \MoveUp, \Forward, \MoveDown}\}^\ast,\;v\in\{\text{\Stopsign}\}\}$
        \item $L(\gamma_7) = \{w \mid |w|_a = 3, |w|_b = 1, w\in \{a,b,c\}^*\}$
    \end{itemize}
    \end{block}
\end{frame}
}

{\setbeamercolor{palette primary}{bg=ExColor}
\begin{frame}{Lösung}
    Alle Lösungen sind Beispiellösungen, es sind auch andere möglich.\\
    Klammern die nicht zur Bedeutung beitragen, dürfen wir für die Kurzschreibweise weglassen.
    \begin{itemize}[<+- | alert@+>]
        \item $\gamma_1 = (aa)^*$
        \item $\gamma_2 = (a)^*(b)^*$
        \item $\gamma_3 = (a|b)^*\ (c|d)$
        \item formal: $\gamma_4 = ((a|b)|c)((a|b)|c)((a|b)|c)$, kurz: $\gamma_4 = (a|b|c)(a|b|c)(a|b|c)$
        \item $\gamma_5 = a(aaa)^*$
        \item formal: $\gamma_6 = ($((\Rewind | \MoveUp) | \Forward) | \MoveDown$)^*$\Stopsign, kurz: $\gamma_6 = ($\Rewind | \MoveUp | \Forward | \MoveDown$)^*$\Stopsign
        \item formal: $\gamma_7 = (c)^*(a(c)^*a(c)^*a(c)^*b\mid a(c)^*a(c)^*b(c)^*a\mid a(c)^*b(c)^*a(c)^*a\mid b(c)^*a(c)^*a(c)^*a)(c)^*$,\\kurz: $\gamma_7 = c^*(ac^*ac^*ac^*b\mid ac^*ac^*bc^*a\mid ac^*bc^*ac^*a\mid bc^*ac^*ac^*a)c^*$ 
    \end{itemize}
\end{frame}
}

{\setbeamercolor{palette primary}{bg=ExColor}
\begin{frame}{Reguläre Ausdrücke}
\begin{columns}
    \column{0.5\textwidth}
\begin{alertblock} {Gegeben sei folgender DEA M:}
    \begin{tikzpicture}[->,>=stealth',shorten >=1pt,auto,node distance=2cm,
                            semithick]
        \node [initial,state]   (0)              {$q_0$};
        \node [state]           (1) [right of=0] {$q_1$};
        \node [state,accepting] (2) [below of=1] {$q_2$};
        \path   (0) edge               node {b} (1)
                    edge               node {a} (2)
                (1) edge               node {b} (2)
                    edge [loop above]  node {a} (1)
                (2) edge [loop below]  node {a,b} (2);
        \end{tikzpicture}
\end{alertblock}
\column{0.5\textwidth}
\metroset{block=fill}
    \begin{block}{Welcher reguläre Ausdruck beschreibt $T(M)$?}
        \begin{enumerate}
            \item \alert<2>{$(a|b(a)^*b)(a|b)^*$}
            \item $a(ab)^*$
            \item $(a|b(a)^*b)(b)^*$
            \item $(a|b)^*$
        \end{enumerate}
    \end{block}
\end{columns}
\end{frame}
}


\begin{frame}[standout]
  Murmelpause
\end{frame}

\section{Wiederholung}
\begin{frame}[fragile]{Das können wir jetzt beantworten}
	\begin{alertblock}{Tag 1: Mengen}
		\begin{itemize}
			\item Was ist eine Menge?
			\item Wie kann man zwei Mengen verknüpfen?
			\item Wie schreibt man formal Mengen auf?
		\end{itemize}
	\end{alertblock}
\end{frame}

\begin{frame}[fragile]{Das können wir jetzt beantworten}
	\begin{alertblock}{Tag 1: Formale Sprachen}
		\begin{itemize}
			\item Was ist eine Formale Sprache?
			\item Was ist ein Alphabet?
			\item Wie zeigt man, dass zwei Sprachen äquivalent sind?
		\end{itemize}
	\end{alertblock}
\end{frame}

\begin{frame}[fragile]{Das können wir jetzt beantworten}
    \begin{alertblock}{Tag 2: Beweise}
    \begin{itemize}
        \item Was ist ein direkter Beweis?
        \item Wie funktioniert die Kontraposition?
        \item Wie funktioniert ein Widerspruchsbeweis?
        \item Wie funktioniert Induktion?
    \end{itemize}
    \end{alertblock}
\end{frame}

\begin{frame}[fragile]{Das können wir jetzt beantworten}
    \begin{alertblock}{Tag 3: Grammatiken}
    \begin{itemize}
        \item Was sind Grammatiken?
        \item Was ist der Zusammenhang zwischen Grammatiken und Sprachen?
        \item Wie finde ich raus, ob ein Wort von einer Grammatik erzeugt wird?
    \end{itemize}
    \end{alertblock}
\end{frame}

\begin{frame}[fragile]{Das können wir jetzt beantworten}
	\begin{alertblock}{Tag 4: Reguläre Grammatiken}
		\begin{itemize}
			\item Wie sehen Produktionsregeln für reguläre Grammatiken aus?
			\item Bilden einer regulären Grammatik für gegebene reguläre Sprache
		\end{itemize}
	\end{alertblock}
\end{frame}

\begin{frame}[fragile]{Das können wir jetzt beantworten}
	\begin{alertblock}{Tag 4: Automaten}
		\begin{itemize}
			\item Was sind Automaten?
			\item Was macht einen deterministischen Automaten aus?
			\item Finden eines (deterministischen) Automaten für gegebene Sprache
		\end{itemize}
	\end{alertblock}
\end{frame}

\begin{frame}[fragile]{Das können wir jetzt beantworten}
	\begin{alertblock}{Heute: Repräsentationen regulärer Sprachen}
		\begin{itemize}
			\item Welche Möglichkeiten gibt es, reguläre Sprachen zu beschreiben?
			\item Wie wandelt man Automaten zu einer äquivalenten Grammatik um?
			\item Was ist ein regulärer Ausdruck?
		\end{itemize}
	\end{alertblock}
\end{frame}

\begin{frame}[fragile]{Das können wir jetzt beantworten}
	\begin{alertblock}{Heute: Reguläre Ausdrücke}
		\begin{itemize}
			\item Wie funktioniert die Konkatenation?
			\item Was bedeuten ($\alpha \mid \beta$) und $(\alpha)^*$?
			\item Finden eines regulären Ausdrucks für gegebene reguläre Sprache
		\end{itemize}
	\end{alertblock}
\end{frame}

\begin{frame}[standout]
  Noch Fragen?
\end{frame}

\begin{frame}[fragile]{Glossar}
    \small
    \begin{tabular}{p{0.2\textwidth} p{0.25\textwidth} p{0.4\textwidth}}
    \toprule
    Abk.&Bedeutung&Was?!\\
    \midrule
       \begin{tikzpicture}[->,>=stealth',shorten >=1pt,auto,node distance=1cm,semithick,baseline=(q0.base)]
        \node[initial,state](q0){$q_0$};
        \end{tikzpicture} & Startzustand & Hier fängt der Automat beim Lesen eines Wortes an \\
        \begin{tikzpicture}[->,>=stealth',shorten >=1pt,auto,node distance=1.4cm,semithick,baseline=(qi.base)]
        \node[state](qi){$q_i$};
        \node[state](qj)[right of=qi]{$q_j$};
        \path (qi) edge node {$a$} (qj);
        \end{tikzpicture} & Zustandsübergang & gibt an, welches Symbol eingelesen werden kann, um in den Folgezustand zu übergehen. \\
        \begin{tikzpicture}[->,>=stealth',shorten >=1pt,auto,node distance=1cm,semithick,baseline=(qe.base)]
        \node[accepting,state](qe){$q_E$};
        \end{tikzpicture} & Endzustand & Hier kann ein fertig gelesenes Wort akzeptiert werden. \\
        \begin{tikzpicture}[->,>=stealth',shorten >=1pt,auto,node distance=2cm,semithick,baseline=(qi.base)]
        \node[state](qi){$\emptyset$};
        \path (qi) edge [loop right] node {$x \in \Sigma$} (B);
        \end{tikzpicture} & Fangzustand & wird benötigt, um Determinismus zu gewährleisten. In Graphiken oft nicht eingezeichnet, ist aber da. Malt den hin. \\
    \bottomrule
    \end{tabular}
\end{frame}

\begin{frame}[fragile]{Glossar}
    \small
    \begin{tabular}{p{0.05\textwidth} p{0.37\textwidth} p{0.43\textwidth}}
    \toprule
    Abk. & Bedeutung & Was?! \\
    \midrule
       T(M) & Sprache von Automat M & Die Sprache, die von einem Automat M erkannt wird \\
       L(G) & Sprache von Grammatik G & Die Sprache, die von einer Grammatik G erzeugt wird \\
       $\gamma$ & kleines Gamma & oft Bezeichner für regulären Ausdruck \\
       L($\gamma$) & Sprache von reg. Ausdruck $\gamma$ & Die Sprache. die von einem regulären Ausdruck $\gamma$ erkannt wird \\
    \bottomrule
    \end{tabular}
\end{frame}
\end{document}
